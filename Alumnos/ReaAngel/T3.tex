\documentclass[letterpaper]{article}
\usepackage[spanish]{babel}
\usepackage[utf8]{inputenc}
\usepackage[top=1.2in, left=0.9in, bottom=1.2in, right=0.9in]{geometry} % sets the margins
\usepackage{amsmath,amsthm,amssymb}
\usepackage{url} % fixes url problem
\usepackage{csquotes}% Recommended
\usepackage[doublespacing]{setspace} % turns on double spacing

\usepackage{graphicx}
\graphicspath{ {d:/Documentos Main/BaseDD/lateximgsh/} }

\title{Permisos en un Usuario SQL}
\author{Rea Aparicio Angel David}
\date{11 Marzo 2021}

\begin{document}
\maketitle

\section{Investigación}
Se vio en clase la forma de crear un usuario con contraseña con el comando \textbf{CREATE USER} [nombre usuario] \textbf{WITH PASSWORD} '[contraseña]'; para agregar un tiempo limite en el que se puede usar esa contraseña, se utiliza el comando: \textbf{VALID-UNTIL} [AAAA-MM-DD] para establecer año, mes y día en el que expira la contraseña.
\section{Capturas de Pantalla}
\begin{itemize}
    \item Creando usuarios de Prueba\par
    \includegraphics[scale=0.9]{usuarios_prueba} 
    Estos usuarios se usaran para revisar el límite de conexiones.
    \item Usuario Especificado \par 
    \includegraphics[scale=0.8]{usuarioesp} 
    \item Creación Rol \par
    \includegraphics[scale=0.8]{creacionrol.png}
    \item Creación Tabla y Asignación de Permisos al Rol \par 
    \includegraphics[scale=0.8]{creaciont_asignacionp.png}
    \item Asignación de Rol a Usuario \par
    \includegraphics[scale=0.8]{asig_rol_to_user.png}
    \item Comprobación de creación de Tabla ingresando como el usuario Especificado \par
    \includegraphics[scale=0.8]{comp_tabla.png}
    \item Comprobación funcionamiento \par
    \includegraphics[scale=0.8]{comp_fun.png}
\end{itemize}
\section{Usuarios} \par
\includegraphics[scale=0.8]{edo_user.png}\par
\section{Pruebas de Conexión} \par
\includegraphics[scale=0.8]{conxinit.png} \par
\includegraphics[scale=0.8]{conxdout.png} \par

Como se puede observar, a pesar de que se limito la conexión a 3 para el usuario oreshi, pudo conectarse mas de 4 veces por lo que CONNECTION LIMIT no funcionó como se esperaba.

\begin{thebibliography}{4}
\bibitem{PostSQL}
Postgres.
\textit{Postgres SQL Documentation.}
\url{https://www.postgresql.org/docs/9.0/sql-grant.html}

\bibitem{PostSQL2}
Postgres.
\textit{Postgres SQL Documentation.}
\url{https://www.postgresql.org/docs/9.2/app-psql.html}

\bibitem{msoft}
Microsoft.
\textit{SQL Documentation}
\url{https://docs.microsoft.com/en-us/sql/t-sql/statements/create-user-transact-sql?view=sql-server-ver15}

\bibitem{sqlt}
SQL Tutotial.
\textit{SQL List of All Tables}
\url{https://www.sqltutorial.org/sql-list-all-tables/}
\end{thebibliography}

\end{document}













