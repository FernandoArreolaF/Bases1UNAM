\documentclass[letterpaper]{article}
\usepackage[english]{babel}
\usepackage[utf8]{inputenc}
\usepackage[top=1.2in, left=0.9in, bottom=1.2in, right=0.9in]{geometry} % sets the margins
\usepackage{amsmath,amsthm,amssymb}
\usepackage{url} % fixes url problem
\usepackage{csquotes}% Recommended
\usepackage[doublespacing]{setspace} % turns on double spacing

\title{Investigación}
\author{Rea Aparicio Angel David}
\date{18 marzo 2021}

\begin{document}
\maketitle

\section*{Dependencia e independencia de la existencia}
Es la condición que tienen los datos de una tabla; si una existencia es independiente de su tabla, este se puede copiar a otras tablas conservando sus atributos, pero si la existencia es dependiente de la tabla en que esta contenido entonces perderá sus atributos al ser copiado a otra tabla o entraran en conflicto con los de la nueva tabla.
\section*{¿Qué es una entidad débil?}
Son las entidades que no tienen un atributo llave propio. Los registros de una entidad débil son identificados por estar relacionados con registros de otra entidad.
Son entidades cuya existencia depende de alguna otra entidad.
\section*{Referencias}
ITCA FEPADE \textit{Entidades Débiles} en \url{https://virtual.itca.edu.sv/Mediadores/dbd/u1/entidades_dbiles.html}

\end{document}













