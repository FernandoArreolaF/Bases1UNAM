\documentclass{article}
\begin{document}
\title{Homework 5: Third Normal Form (3NF)}
\author{Daniel Alberto Zarco Manzanares}
\date{\today}
\maketitle
\paragraph{Employee's exercise\\\\}
Third normal form from exercise about employee with his store's information that contain worker's view.
\paragraph{First table\\\\}
\begin{tabular}{|c|c|c|c|c|}
Staff's number & Name & Position & Salary & Branch's number\\
\hline
S1550 & Tom Daniels & Manager & 46000 & B001\\
S0003 & Sally Adams & Assistant & 30000 & B001\\
S0010 & Mary Martinez & Manager & 50000 & B002\\
S3250 & Robert Chin & Supervisor & 32000 & B002\\
S2250 & Sally Stern & Manager & 48000 & B004\\
S0415 & Art Peter & Manager & 41000 & B003 \\
\end{tabular}
\paragraph{Second table\\\\}
\begin{tabular}{|c|c|c|}
Branch's number & Branch's Address & Telephone's number\\
\hline
B001 & 8 Jefferson Way, Portland, OR 97201 & 503-555-3618\\
B001 & 8 Jefferson Way, Portland, OR 97201 & 503-555-3618\\
B002 & City Center Plaza, Seattle, WA 98122 & 206-555-6756\\
B002 & City Center Plaza, Seattle, WA 98122 & 206-555-6756\\
B004 & 16-14 th Avenue, Seattle, WA 98128 & 206-555-3131\\
B003 & 14-8 th Avenue, New York, NY 10012 & 212-371-3000\\
\end{tabular}
\paragraph{Conclusion\\\\}
These tables are in third normal form from normalization's rules:
\begin{itemize}
\item 1NF) All elements are atomic and don't have multi-value's attributes, and we don't have repeated row in the tables.
\item 2NF) We don't have partial relations of form x -> y, y -> z.
\item 3NF) Tables don't have transition relations, and tables are in second normal form.
\end{itemize}
\newpage
\paragraph{Car sale's exercise\\\\}
Third normal form from exercise about car sale with his store's information that contain sales men view.
\paragraph{Firs table\\\\}
\begin{tabular}{|c|c|c|}
Car's number & Sale date & Discount\\
\hline
\end{tabular}
\paragraph{Second table\\\\}
\begin{tabular}{|c|c|}
Seller & Commission\\
\hline
\end{tabular}
\paragraph{Third table\\\\}
\begin{tabular}{|c|c|}
Car's number & Seller\\
\hline
\end{tabular}
\paragraph{Conclusion\\\\}
These tables are in third normal form from normalization's rules:
\begin{itemize}
\item 1NF) All elements are atomic and don't have multi-value's attributes, and we don't have repeated row in the tables.
\item 2NF) We don't have partial relations of form x -> y, y -> z.
\item 3NF) Tables don't have transition relations, and tables are in second normal form. We follow the implications of tables.
\end{itemize}

\end{document}

