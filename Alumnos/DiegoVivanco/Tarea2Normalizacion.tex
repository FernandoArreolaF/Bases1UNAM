\documentclass[spanish]{article}
\usepackage[T1]{fontenc}
\usepackage[utf8]{inputenc}
\usepackage[spanish]{babel}
%\usepackage[spanish,activeacute,mexico]{babel}
\usepackage[margin=2.5cm]{geometry} %tamaño de la hoja
\usepackage{graphicx} % Paquete para imágenes
\usepackage{amsmath} % Paquete para usar más funciones de matemáticas
\usepackage{amssymb} %Para el therefore
\usepackage{enumitem}

\begin{document}
%Datos del equipo
\title{
\centering{\includegraphics[width=1\linewidth]{escudos.png}\\[1cm]}\\
UNIVERSIDAD NACIONAL AUTONOMA DE MEXICO\\
\vfill
Facultad de Ingenieria\\
\vfill
{\bfseries TAREA 2}
\vfill
{\bfseries Normalización de Bases de Datos }
\vfill
\\Grupo 5\
\vfill
Semestre 2020-2\\
\vfill
BASES DE DATOS
\vfill
Profesor: Ing. Fernando Arreola Franco}
\vfill
% Agreguen sus nombres :v
\author{\textbf{Integrantes}:\\
Vivanco Quintanar, Diego Armando\\}
\date{23 de marzo de 2020}
\maketitle
\newpage


\section{EJERCICIO}

    A partir de la siguiente tabla normalizar a la primera forma.
    
     \begin{table}[ht]
    \centering
	\begin{tabular}{|c|c|c| c|l|}
	\hline
	\textbf{EmployeeID} & \textbf{Name} & \textbf{project} &  \textbf{time}\\ \hline
	 EN1-26 & Sean O'Brien & 30-452-T3, 30-457-T3, 32-244-T3 & 0.25,0.40, 0.30\\ \hline
	 EN1-33 & Amy Guya & 30-452-T3,30-382-T3,32-244-T3 & 0.05,0.35, 0.60\\ \hline
	 EN1-35 & Steven  & 30-452-T3, 31-238-TC & 0.15,0.80\\ \hline
	 EN1-36 & Elizabeth Roslyn & 35-152-TC & 0.90\\ \hline
	 EN1-38 & Carol Schaaf & 36-272-TC & 0.75\\ \hline
	 EN1-40 & Alexandra Wing & 31-238-TC, 31-241-TC & 0.20,0.70 \\ \hline
	\end{tabular}
	\caption{Tabla original, con el ID del empleado, nombre del empleado, proyecto en el que trabaja el empleado y el tiempo como atributos aun sin normalizar.} \label{semi}
	\end{table}
	
	
	La 1FN prohibe los grupos repetidos, si observamos hay grupos repetidos en base al nombre del empleado, para evitar esto se opta por crear dos tablas que tengan en comun la llave primaria que es el ID del empleado, esto nos permitira buscar la informacion de los empleados sin tener grupos repetidos.\\
	Asi pues las tablas quedan normalizadas a la 1FN de la siguiente manera:\\
    
    
    
    
    
     \begin{table}[ht]
    \centering
	\begin{tabular}{|c|c|l|}
	\hline
	\textbf{EmployeeID} & \textbf{Name}\\ \hline
	 EN1-26 & Sean O'Brien\\ \hline
	 EN1-33 & Amy Guya\\ \hline
	 EN1-35 & Steven Baranco\\ \hline
	 EN1-36 & Elizabeth Roslyn\\ \hline
	 EN1-38 & Carol Schaaf \\ \hline
	 EN1-40 & Alexandra Wing \\ \hline
	\end{tabular}
	\caption{Tabla 1, con el ID del empleado y el nombre del empleado como atributos.} \label{semi}
	\end{table}
	
	
	
    
    
       
    
     \begin{table}[ht]
    \centering
	\begin{tabular}{|c|c|l|}
	\hline
	\textbf{EmployeeID} & \textbf{Project} & \textbf{Time}\\ \hline
	 EN1-26 & 30-452-T3 & 0.25\\ \hline
	 EN1-26 & 30-457-T3 & 0.40\\ \hline
	 EN1-26 & 32-482-T3 & 0.30\\ \hline
	 EN1-33 & 30-452-T3 & 0.05\\ \hline
	 EN1-33 & 30-482-T3 & 0.35\\ \hline
	 EN1-33 & 32-244-T3 & 0.60\\ \hline
	 EN1-35 & 30-452-T3 & 0.15\\ \hline
	 EN1-35 & 31-238-TC & 0.80\\ \hline
	 EN1-36 & 35-152-TC & 0.90\\ \hline
	 EN1-38 & 36-272-TC & 0.75\\ \hline
	 EN1-40 & 31-238-TC & 0.20\\ \hline
	 EN1-40 & 31-241-TC & 0.70\\ \hline
	\end{tabular}
	\caption{Tabla 2, con el ID del empleado, el proyecto en el que participa el empleado y el tiempo.} \label{semi}
	\end{table}
	


\end{document}

