\documentclass{article}
\usepackage[utf8]{inputenc}
\usepackage{graphicx}

\begin{document}
	
	\begin{titlepage}
		\centering
		\vspace*{1cm}
		\Large Universidad Nacional Autónoma de México\\
		\Large Facultad de Ingeniería\\
		\vspace*{0.5cm}
		\large Nombre del Alumno:\\
		\large Silverio Martínez Andrés \\
		\vspace*{0.5cm}
		\large Nombre de la Materia:\\
		\large Bases de Datos\\
		\vspace*{0.5cm}
		\large Nombre del Profesor:\\
		\large Fernando Arreola Franco \\
		\vspace*{0.5cm}
		\large Tarea 4:\\
		\large  Dependencia e independencia de existencia y entidad débil\\
		\vspace*{0.5cm}
		\large Grupo:\\
		\large 1 \\
		\vspace*{0.5cm}
		\large Semestre:\\
		\large 2024 - 2:\\
		\vspace*{0.5cm}

	\end{titlepage}
	
	\title{Tarea 4}
	\author{Silverio Martínez Andrés}
	\date{\today}
	
	\maketitle
	
	\section{Dependencia e independencia de existencia}
	Una dependencia en existencia se refiere a una relación entre entidades, en la que la existencia de una entidad, si o sí, va a depender de la exitencia de la otra. Por ejemplo, en un modelo de base de datos, se puede tener una entidad “Pedido” y una entidad “Producto”. Si cada pedido debe contener al menos un producto, entonces la entidad “Pedido” tiene una dependencia en existencia con la entidad “Producto”.
	
	\vspace*{0.5cm}
	
	Mientras que la independencia de datos es la característica de una base de datos, la cual permite modificar la estructura o el formato de los datos en un área sin afectar a los datos de otras áreas. Existen dos tipos, que son:
	
	\vspace*{0.5cm}
	
	- Independencia lógica: Permite realizar cambios en la estructura de los datos de manera independiente, o modificar las aplicaciones que utilizan esos datos.

	- Independencia física: Permite realizar cambios en la estructura física de la base de datos sin la necesidad de realizar cambios en las aplicaciones que utilizan las bases de datos.

	\vspace*{5 cm}

	\section{Entidad débil}
	Una entidad débil es una entidad que cuyas propiedades o atributos no la identifican completamente, solamente de forma parcial. Esta entidad debe de ser partícipe de una relación que la ayude a identificarla.
	En el modelo E/R una entidad débil se representa con el nombre de la entidad encerrado en un rectángulo doble, como se muestra a continuación:

	
\begin{figure}[h]
	\centering
	\includegraphics[width=0.5\linewidth]{screenshot001}
	\caption{Entidad débil}
	\label{fig:screenshot001}
\end{figure}

	\section{Referencias}

[1]	3.1.- Tipos: fuertes y débiles. | DAM\_BD03\_Contenido”. S'Arreplec. Accedido el 15 de febrero de 2024. [En línea]. Disponible: https://sarreplec.caib.es/pluginfile
.php/10828/mod\_resource/content/2/31\_tipos\_fuertes\_y\_dbiles.html

\vspace{0.5 cm}

[2]	¿Qué es la Independencia de Datos?” Ordenadores y Portátiles. Accedido el 15 de febrero de 2024. [En línea]. Disponible: https://ordenadores-y-portatiles.com/independencia-de-datos/
	

\end{document}